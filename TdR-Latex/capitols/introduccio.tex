\section*{Justificació del treball}
\addcontentsline{toc}{section}{Justificació del treball}
Vivim en una societat en la qual utilitzem dispositius electrònics per a gairebé tot; per a comunicar-nos, treballar, oci, estudiar, jugar, comprar... 

Les empreses del sector informàtic pretenen millorar aquests dispositius per fer-los més segurs, estables, ràpids, eficients... per destacar-se de la competència i guanyar quota de mercat. 

Hi ha diverses formes de millorar els dispositius. Utilitzant millors components s'obtindrà un millor dispositiu. També és molt important la programació d'aquest, tant del sistema operatiu com de les aplicacions. No serveix de res tenir el millor maquinari (hardware)  si a l'hora de realitzar qualsevol tasca el programari (software) no és eficient.

Per això, aquest treball se centra a analitzar una petita part del programari i implementar-lo, amb la intenció de mesurar teòricament i pràcticament el cost d'aquests procediments per poder comparar-los i determinar que tan eficients són. Analitzar l'eficiència dels procediments serveix per predir quin resoldrà una tasca més eficientment i també per optimitzar-los. Finalment, el resultat d'optimitzar el programari serà un dispositiu més eficient, i, per tant, més ràpid.

Personalment, des de fa un parell d'anys participo en concursos de programació, aquests consisteixen a resoldre problemes de matemàtiques en què la solució és un programa. Aquest ha de resoldre el problema de la forma més eficient possible, i gairebé sempre s'utilitzen algoritmes.

En aquest treball ens centrarem en la part d'analitzar i programar alguns algoritmes, i per simplificar-ho i entendre bé els conceptes, utilitzaré problemes dels concursos com a exemples.

\section*{Objectius}
\addcontentsline{toc}{section}{Objectius}
En aquest treball podem definir dos tipus d'objectius, els que fan referència al contingut del treball:
\begin{itemize}
    \item Estudiar i definir què és l'algorísmia i per a què serveix. 
    \item Estudiar com s'analitza l'eficiència dels algoritmes.
    \item Estudiar de forma teòrica alguns algoritmes.
    \item Realitzar un programa per visualitzar els algoritmes estudiats.
    \item Implementar aquells algoritmes i comparar els resultats pràctics amb els teòrics.
\end{itemize}
I els que fan referència a la realització del treball:
\begin{itemize}
    \item Programar el treball en \LaTeX. \\
    - LaTeX és un sistema de composició de textos, orientat especialment a la creació de llibres, documents científics i tècnics que continguin fórmules matemàtiques. 
\end{itemize}

\section*{Metodologia}
\addcontentsline{toc}{section}{Metodologia}
Per assolir els nostres objectius, partirem el treball en dues parts teòriques i dues parts pràctiques.

Primer definirem alguns conceptes bàsics, explicarem que són els algoritmes, els programes, com s'analitzen i perquè és important fer-ho.

Tot seguit, veurem exemples més concrets d'algoritmes i els analitzarem com haurem explicat al capítol 1.

A continuació, programarem els exemples que haurem analitzat al capítol 2 i compararem els resultats teòrics (capítol 2) amb els resultats pràctics (capítol 3). També he implementat una visualització per entendre bé el funcionament d'aquests algoritmes.

Per últim, explicarem breument que és \LaTeX \space, quines diferències hi ha entre un processador de textos i aquest llenguatge de programació, i perquè l'he utilitzat per redactar aquest treball.