\section*{Justificació del treball}
\addcontentsline{toc}{section}{Justificació del treball}
Vivim en una societat en la qual utilitzem dispositius electrònics per a tot. Per a comunicar-nos, treballar, oci, estudiar, jugar, comprar... I quan hi ha molta demanda d'aquests dispositius, es busca optimitzar-los, fer-los més agradables per l'usuari i fer-los més ràpids.

Estar clar que és més còmode esperar menys a què s'encengui l'ordinador, obrir més ràpidament les aplicacions del mòbil, i, per tant, consumir menys bateria. O buscar un fitxer i trobar-lo en segons en lloc de minuts, actualitzar el sistema en menys temps, fer una cerca a internet i tardar mil·lèsimes de segon... Si el treball de l'usuari es troba interromput a causa del dispositiu, està clar que l'usuari no haurà tingut una bona experiència i evitarà fer ús d'aquesta eina.

Hi ha diverses formes de fer un dispositiu més ràpid, òbviament utilitzant millors components tindrem un millor dispositiu, però el preu també serà molt superior. També és molt important la programació del dispositiu, tant del sistema operatiu com de les aplicacions. No serveix de res tenir el millor dispositiu si a l'hora de fer la feina el programa no és eficient. No s'estaria utilitzant el màxim potencial dels components. A més, té el mateix cost programar-ho millor o pitjor, per això és tan important optimitzar-ho al màxim per crear un dispositiu eficient, de qualitat i assequible.

Personalment, des de fa un parell d'anys participo en concursos de programació, aquests consisteixen en resoldre problemes de matemàtiques en què la solució és un programa. Aquesta solució ha de resoldre el problema de la forma més eficient possible, i sempre s'utilitzen algoritmes.

En aquest treball ens centrarem en la part de programar de forma eficient, i per simplificar-ho i entendre bé els conceptes, utilitzaré alguns problemes dels concursos com a exemples.

\section*{Objectius}
\addcontentsline{toc}{section}{Objectius}
En aquest projecte podem definir dos tipus d'objectius, els que fan referència al contingut del treball:
\begin{itemize}
    \item Estudiar i definir que és l'algorísmia i per a què serveix. 
    \item Estudiar com s'analitza i compara l'eficiència dels algoritmes i programes.
    \item Estudiar i comparar de forma teòrica alguns algoritmes.
    \item Implementar aquells algoritmes i comparar de forma pràctica els programes.
    \item Realitzar un programa per visualitzar els algoritmes estudiats.
\end{itemize}
I els que fan referència a la realització del treball:
\begin{itemize}
    \item Realitzar un treball amb el programari Overleaf i \LaTeX.
\end{itemize}

LaTeX és un sistema de composició de textos, orientat especialment a la creació de llibres, documents científics i tècnics que continguin fórmules matemàtiques. 

Overleaf és un editor en línia de \LaTeX. 


\section*{Metodologia}
\addcontentsline{toc}{section}{Metodologia}
Per assolir els nostres objectius, partirem el treball en dues parts teòriques i dues parts pràctiques: En la primera part teòrica, definirem tots els conceptes importants, explicarem que és un algoritme, per què són importants, i com s'analitzen per poder determinar quin és l'òptim. 

En la segona part teòrica posarem exemples més concrets i els analitzarem per poder predir el seu comportament en diferents situacions. 

En la primera part pràctica implementarem els algoritmes de la segona part teòrica, i compararem els resultats teòrics amb els pràctics. També implementarem una visualització dels algoritmes.

I finalment, acabarem editant aquest treball en el llenguatge \LaTeX.