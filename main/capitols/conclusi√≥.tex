Finalment, he assolit els objectius plantejats a l'inici d'aquest treball:

\quad \textbullet\  Diferenciar els algoritmes dels programes.

En el capítol 1 hem definit els conceptes més importants per entendre bé què són els algoritmes i els programes. També hem posat alguns exemples.

\quad \textbullet\ Estudiar com s'analitza i expressa l'eficiència dels algoritmes.

Al final del capítol 1 he explicat per què és tan important analitzar l'eficiència dels algoritmes, i com es pot fer. He definit la notació que s'acostuma a utilitzar i he posat alguns exemples. També he comparat les complexitats més bàsiques per veure que efectivament hi ha molta diferència de creixement entre algunes complexitats. Finalment, he explicat com s'analitzen algoritmes en estructures de dades no lineals.

A l'hora de definir la notació de Landau, m'he trobat algunes dificultats. No tenia molt clar si definir el concepte amb la definició matemàtica o amb exemples de programació. Finalment, em vaig decantar per una definició formal, tot i que em va costar trobar i entendre la informació. 

\newpage
\quad \textbullet\ Estudiar de forma teòrica alguns algoritmes.

En el capítol 2 he explicat quatre algoritmes. Dos dels quals resolen un problema de cerca i els altres dos d'ordenació. En aquest capítol hem vist el funcionament d'aquests algoritmes i també els hem analitzat. En analitzar-los, hem pogut determinar quin dels dos és més eficient en termes d'eficiència temporal.

\quad \textbullet\ Realitzar un programa per visualitzar els algoritmes estudiats.

He pogut implementar el programa sense gaires dificultats. Com que és un programa llarg i format de diversos programes en diferents fitxers que s'acaben ajuntant en un, hi ha la dificultat d'estructurar bé el programa per fer més fàcil la lectura i poder solucionar errors i depuracions amb més facilitat.

\quad \textbullet\ Implementar aquells algoritmes i comparar els resultats pràctics amb els teòrics.

En el capítol 3 he implementat els algoritmes que hem estudiat al capítol 2 i he fet gràfics per comprovar si l'anàlisi pràctica i teòrica coincideixen. 

En aquest capítol m'he trobat amb algunes dificultats. A l'hora de fer els gràfics en alguns casos hi havia dades que no coincidien amb els resultats esperats, així que vaig analitzar per què passava això, i vaig arribar a la conclusió que l'ordinador estava executant algun procés en segon pla. Inicialment, executava 10 vegades seguides cada mida d'entrada, això provocava que quan hi havia un procés en segon pla les 10 mostres i les mides d'entrada del voltant es veien afectades. I encara que fes la mitjana, totes les mostres de la mateixa mida d'entrada estaven afectades. Així que vaig repetir els gràfics, però aquesta vegada executant una vegada totes les mides d'entrada i ho vaig repetir 10 vegades. D'aquesta manera, només canviant l'ordre en què s'executen les mostres, en cas que s'executi algun procés en segon pla, només afectarà una mostra de mides d'entrada diferents i a l'hora de fer la mitjana, afecta molt menys el resultat.

D'aquesta manera vaig aconseguir uns gràfics molt acurats.

\newpage
\quad \textbullet\ Programar el treball en \LaTeX. 

Com hem explicat al capítol 4, \LaTeX \space és un llenguatge de programació i requereix temps aprendre'l.

Per fer aquest treball, primer vaig fer tot el format del document: capçalera, marges, tipus de lletra, espaiat, peus de pàgina, numeració de pàgina, l'estil dels enllaços, l'estil dels gràfics, taules, figures... També vaig crear un document diferent per a cada capítol per importar-los tots al document principal. I després de crear tota l'estructura del treball, vaig començar a redactar el contingut.

Va ser difícil fer l'estructura del document, però considero que ha valgut la pena. Perquè a l'hora d'escriure el contingut m'ha resultat més senzill que amb qualsevol altre editor de text, i el resultat considero que és molt bo.

En conclusió, he assolit tots els objectius que m'he marcat en aquest treball. Encara que m'hauria agradat analitzar algoritmes més complexos i que tinguessin una aplicació pràctica més directa i interessant, com alguns algoritmes de grafs o conceptes com la programació dinàmica. Però, he prioritzat explicar les nocions bàsiques per fer més fàcil l'enteniment del treball i a causa del límit de pàgines no he pogut arribar a explicar alguns conceptes i algoritmes que m'hagués agradat.

Finalment, vull agrair la feina del meu tutor Jordi Irazuzta per totes les propostes i idees que ha aportat en aquest treball i tot el temps que ha dedicat.

També vull agrair a la família pel suport que m'han donat, especialment al meu pare per ajudar-me a corregir gran part de les faltes ortogràfiques i gramaticals. 
